\documentclass{article}
\usepackage{amsmath}
\usepackage{amsfonts}
\usepackage{amssymb}
\usepackage{stmaryrd}

\DeclareMathOperator*{\fl}{fl}
\DeclareMathOperator*{\round}{round}

% "binade" - a "Binary decade".  Binade floor: the smallest power of two less
% than or equal to the given number
\newcommand{\binadefloor}[1]{\llfloor #1 \rrfloor}

\begin{document}

\section{The Problem}

We want to interpolate linearly between two floats $a$ and $b$, with $n$
steps. In other words, compute the function
\begin{equation*}
  \rho_i = a \frac{n-i}{n-1} + b \frac{i-1}{n-1}
\end{equation*}
at $i = 1,\ldots, n$.  For numerical robustness, we require that our
numerical approximation $\bar\rho_i$ exactly hits the end points, that is,
$\bar\rho_1 = a$ and $\bar\rho_n = b$.  Ideally we also want it to be
correctly rounded in the interior of the interval, but may trade off some
accuracy for speed.


\section{Notation}

We define:
\begin{itemize}
\item $p$: the precision of the floating point format in bits (i.e. $p=53$ for IEEE binary64).  We also assume that $n \leq 2^p$.
\item $r$: an integer such that $|i - i_0| < 2^r$. At worst, we can take $r = p$. 
  (FIXME)
\end{itemize}

We denote twice-precision numbers with a tilde. If $\tilde x$ is twice
precision, we write the floating point parts containing the high and low bits
as $\tilde x_h$ and $\tilde x_l$.

We write
\begin{equation*}
  \binadefloor{x}
\end{equation*}
for the smallest element of the binade containing $x$, i.e. largest integer
power of $2$ such that $2^u \leq |x|$.


\section{Algorithm}

Our approach is to construct an object containing:
\begin{itemize}
\item $n$: the length
\item $i_0$: the index of the element closest to zero.
\item $\tilde z$: $\rho_{i_0}$ stored in twice precision (``double-double'' format)
\item $\tilde s$: an approximation of the step size $s = (b-a)/(n-1)$
  stored in a twice precision \emph{layout}, but at reduced precision.
\end{itemize}

Now we need to compute
\begin{equation*}
  \bar\rho_i = \tilde z + \tilde s \times (i - i_0)
\end{equation*}

There are 4 sources of error:
\begin{enumerate}
\item $\tilde z$ as an approximation of $\rho_{i_0}$
\item $\tilde s$ as an approximation of $s$
\item the operation $ \tilde y = \tilde s \times (i-i_0)$.
\item the operation $ \bar\rho_i = \tilde z + \tilde y$.
\end{enumerate}

\subsection{Calculating $\tilde z$}
\label{sec:calculating-tilde-z}

If $a$ and $b$ have the same signs, we choose $z$ (and hence $\tilde z$) to be the one with the smaller magnitude. This is obviously exact.

Otherwise, we calculate $\tilde z$ as follows:

\begin{equation*}
  \tilde z = \frac{a \times (n-i_0) + b \times (i_0 - 1)}{n-1}
\end{equation*}
The multiplications are performed using twice precision beginning with numbers 
in once precision, so are exact. Call the intermediate results $A$ and $B$
respectively.

Now I'm pretty sure that $1/2 \leq |A/B| \leq 2$, which would imply that the 
result is exact.  Moreover, this implies that $A_h + B_h$ is also exact, so
will save us some operations.

\textbf{CJF - this is not the case; consider $n=2$ for example, and put $A$ nearer the origin.}

Hence the only error is incurred in the final division, hence

\begin{equation*}
|\epsilon_z| \leq \binadefloor{z} \times 2^{-2p} 
\end{equation*}
(TODO: figure this out exactly)



\subsection{Calculating $\tilde s$}
\label{sec:calculating-tilde-s}

We store $\tilde s$ so that $\tilde s_h$ with only $q = \max(1, p-r)$ significant bits (i.e. the remaining $p-q$ bits are zero). This ensures that the operation
\begin{equation*}
  \tilde s_h \times (i - i_0)
\end{equation*}
is exact. Moreover, we assume that $\tilde s_h$ is computed optimally, then
\begin{equation*}
  |s - \tilde s_h| \leq \binadefloor{s} \times 2^{-q}
\end{equation*}

This implies that
\begin{equation*}
  |s - \tilde s| \leq \binadefloor{s} \times 2^{-q-p-2} \leq \binadefloor{s} \times \frac{1}{8}
\end{equation*}


\subsection{Multiplication}
\label{sec:multiplication}


\subsection{Addition}
\label{sec:addition}

\begin{equation*}
  |(\tilde z \boxplus \tilde y) - (x + y) | \leq |(\tilde x \boxplus \tilde y) - (\tilde x + \tilde y)| + \epsilon_z + \epsilon_y
\end{equation*}


Ideally, we would calculate $i_0$ as
\begin{equation*}
  i_0 = \round \left(\frac{a n - b }{a-b} \right)
\end{equation*}

Let's say we can compute $i_0$ such that
\begin{equation*}
  |\rho_i| \geq \frac{1}{2} |\rho_{i_0}| \quad \forall i
\end{equation*}



%$|\bar\rho_i - \rho_i | \leq 0.7 \ulp(\rho)$??




\end{document}



%%% Local Variables:
%%% mode: latex
%%% TeX-master: t
%%% End:
